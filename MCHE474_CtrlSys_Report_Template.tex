\documentclass[12pt]{article}
\usepackage{times}
\usepackage[english]{babel}
\usepackage[utf8x]{inputenc}
\usepackage[colorinlistoftodos]{todonotes}
\usepackage[margin=1in]{geometry}
\usepackage{graphicx}
\usepackage{epstopdf}
\usepackage{cite}
\usepackage{listings}
\usepackage{dtklogos}
\usepackage{wrapfig}
\usepackage{subfigure}
\usepackage{amsmath}
\usepackage{amsthm}
\usepackage{amssymb}
\usepackage{amscd}
\usepackage{caption}
\usepackage{etoolbox}
\usepackage{fancyhdr}
\usepackage{stackengine}
\usepackage[export]{adjustbox}
\patchcmd{\thebibliography}{\section*{\refname}}{}{}{}
\usepackage[document]{ragged2e}    %This causes text to left align
\usepackage[colorlinks=true, linkcolor=black,citecolor=black,urlcolor=blue]{hyperref}
\bibliographystyle{IEEEtran}
\DeclareGraphicsRule{.tif}{png}{.png}{`convert #1 `dirname #1`/`basename #1 .tif`.png}

\title{MCHE 474: Lab 1}

\begin{document}
\lefthyphenmin3
\righthyphenmin4
% \pretolerance=2000
% \tolerance=500 
% \emergencystretch=10pt
%\raggedright     %Stops LaTeX from automatically hyphenating the right margin to fit better
%Combine this with \usepackage[document]{ragged2e} to get a text align left similar to natural MS Word


%-------------------------------------------------------------
%Start of Paper
%-------------------------------------------------------------

%%%%%%%%%%%%%%%%%%%%%%%%%%%%%%%%%%%%%%%%%%%%%%%%%%%%%%
%%%%%%%%%%%%%%%%%%%%%%% TITLE PAGE %%%%%%%%%%%%%%%%%%%%%%%%
%%%%%%%%%%%%%%%%%%%%%%%%%%%%%%%%%%%%%%%%%%%%%%%%%%%%%%

\begin{titlepage}

\newcommand{\HRule}{\rule{\linewidth}{0.5mm}} % Defines a new command for the horizontal lines, change thickness here

\center % Center everything on the page
 
%----------------------------------------------------------------------------------------
%	Heading Section
%----------------------------------------------------------------------------------------

\textsc{\LARGE University of Louisiana at Lafayette}\\[1.5cm] % Name of your university/college
\textsc{\Large Control Systems}\\[0.5cm] % Major heading such as course name
\textsc{\large MCHE 474}\\[0.5cm] % Minor heading such as course title

%----------------------------------------------------------------------------------------
%	Title Section
%----------------------------------------------------------------------------------------

\HRule \\[0.4cm]
{ \huge \bfseries Lab 1}\\[0.4cm] % Title of your document
\HRule \\[1.5cm]
 
%----------------------------------------------------------------------------------------
%	Author Section
%----------------------------------------------------------------------------------------

\begin{minipage}{0.4\textwidth}
\begin{flushleft} \large
\emph{Author:}\\
\textsc{Matthew J. Begneaud} \\% Your name
\end{flushleft}
\end{minipage}
~
\begin{minipage}{0.4\textwidth}
\begin{flushright} \large
\emph{Professor:} \\
\textsc{Dr. Mostafa A. Elsayed} % Supervisor's Name
\end{flushright}
\end{minipage}\\[2cm]

% If you don't want a supervisor, uncomment the two lines below and remove the section above
%\Large \emph{Author:}\\
%John \textsc{Smith}\\[3cm] % Your name

%----------------------------------------------------------------------------------------
%	Date Section
%----------------------------------------------------------------------------------------

{\large \today}\\[3cm] % Date, change the \today to a set date if you want to be precise

%----------------------------------------------------------------------------------------
%	Logo Section
%----------------------------------------------------------------------------------------

\includegraphics[width=4in]{UL_logo.jpg}\\[1cm] % Include a department/university logo - this will require the graphicx package
 
%----------------------------------------------------------------------------------------

\vfill % Fill the rest of the page with whitespace

\end{titlepage}

%%%%%%%%%%%%%%%%%%%%%%%%%%%%%%%%%%%%%%%%%%%%%%%%%%%%%%
%%%%%%%%%%%%%%%%%%%%%%% TABLE OF CONTENTS %%%%%%%%%%%%%%%%%%%
%%%%%%%%%%%%%%%%%%%%%%%%%%%%%%%%%%%%%%%%%%%%%%%%%%%%%%

\tableofcontents

\listoffigures

\newpage

%%%%%%%%%%%%%%%%%%%%%%%%%%%%%%%%%%%%%%%%%%%%%%%%%%%%%%
%%%%%%%%%%%%%%%%%%%%%%% REPORT %%%%%%%%%%%%%%%%%%%%%%%%%%
%%%%%%%%%%%%%%%%%%%%%%%%%%%%%%%%%%%%%%%%%%%%%%%%%%%%%%

\addcontentsline{toc}{section}{Introduction} % Add before each section
\section*{\fontsize{12}{12}\selectfont \large Introduction}




\addcontentsline{toc}{section}{Theory} % Add before each section
\section*{\fontsize{12}{12}\selectfont \large Theory}



\addcontentsline{toc}{section}{Procedure} % Add before each section
\section*{\fontsize{12}{12}\selectfont \large Procedure}




\addcontentsline{toc}{section}{Results and Discussion} % Add before each section
\section*{\fontsize{12}{12}\selectfont \large Results and Discussion}




\addcontentsline{toc}{section}{Conclusion} % Add before each section
\section*{\fontsize{12}{12}\selectfont \large Conclusion}





%\section*{\fontsize{12}{12}\selectfont \large References}

\begin{thebibliography}{2}

% Example
%\bibitem{Wagner}
%Ng, K., Wagner, S.W., Camelio, J., Emblom, W.J. (2010). ?Experimental Analysis of Micro Tube
%Hydroforming Process.? Transactions of NAMRC of SME, 38, 577-584.

\end{thebibliography}



%\section*{\fontsize{12}{12}\selectfont APPENDIX}

%\begin{table}[h!]
%  \caption{}
%  \includegraphics[width=\linewidth]{table1.png}
%\end{table}




\end{document}







----------------------------Templates-------------------------------

-------------------------Figure-----------------------

\begin{figure}[h!]  
  \centering
    \includegraphics[width=\linewidth]{**file**}
    \caption{Docking Station}
\end{figure}

---------------------------Table-----------------------
\begin{table}[ht]
\caption{Nonlinear Model Results} % title of Table
\centering % used for centering table
\begin{tabular}{c c c c} % centered columns (4 columns)
\hline\hline %inserts double horizontal lines
Case & Method\#1 & Method\#2 & Method\#3 \\ [0.5ex] % inserts table
%heading
\hline % inserts single horizontal line
1 & 50 & 837 & 970 \\ % inserting body of the table
2 & 47 & 877 & 230 \\
3 & 31 & 25 & 415 \\
4 & 35 & 144 & 2356 \\
5 & 45 & 300 & 556 \\ [1ex] % [1ex] adds vertical space
\hline %inserts single line
\end{tabular}
\label{table:nonlin} % is used to refer this table in the text
\end{table}



probably best to insert as an image from excel

\bigskip\\
\begin{table}[h!]
  \caption{}
  \includegraphics[width=\linewidth]{**file**}
\end{table}
\bigskip\\





-----------------------------Equations------------------------
-----------------------------Regular
\begin{equation}
a = b + c
\end{equation}

--------------------------------- Multiline
\begin{multline}
a = b + c + d + e + f
+ g + h + i + j \\
+ k + l + m + n + o
\end{multline}

-------------------------------Citations-------------------------
\bibitem{Author last name}
  Last, First., year of publication,
  article name, book(etc) name, from \\
  link goes here

----------------------------------other-----------------------------

equations:
http://moser-isi.ethz.ch/docs/typeset_equations.pdf

citations:
http://library.missouri.edu/engineering/about/guides/asme
https://www.asme.org/shop/proceedings/conference-publications/references
